%% Redéfinition de \maketitle pour y ajouter le genre littéraire
%\makeatletter
%\def\maketitle{
%  \begin{titlepage}
%  \newgeometry{right=1.5in,left=1.5in}
%  \begin{center}
%    {\large\uppercase{Grande œuvre}}
%    \null\vfill
%    {\Large \@author}
%    \vskip 1cm
%    {\LARGE \@title}
%    \vskip 0.5cm
%    {\large{\@genre}}
%    \vskip 0.5cm
%    {\Large \@date}
%  \vfill\null
%
%    \scholarlyversion{{\large Édition savante}
%
%    \footnotesize{Avec préface, notes de bas de page, \xenism{dramatis personæ}, bibliographie, et glossaire.}}%
%  \commonversion{Édition commune}
%  \end{center}
%  \restoregeometry
%  \end{titlepage}
%}
%\def\genre#1{\def\@genre{#1}}
%\def\subtitle#1{\def\@subtitle{#1}}
%\makeatother



\makeatletter
\def\maketitle{
  \begin{titlepage}
  \pagecolor{creme}
  \newgeometry{right=0in,left=0in, top=26mm, bottom=0mm}
  \begin{center}
    \includegraphics[height=57mm]{coeur-poignarde.pdf} 
    \vskip 8mm
    {\fontsize{24}{0}\selectfont \color{rouge}\uppercase{D’Amour \emph{\&} de Guerre}}
    \vskip 7mm
    {\fontsize{15}{0}\selectfont \color{noir}{Recueil de poésie}}
    \vskip 8mm
    {\fontsize{16}{0}\selectfont \color{noir}\includegraphics[height=0.9em]{logo-gris-sur-rouge.pdf}}
    \vskip 67mm
    {\fontsize{15}{0}\selectfont \color{noir}{\fontfamily{cmdh}\pique}}
%   \vskip -0.7mm
%   \textcolor{noir}{\rule{1em}{0.2pt}}
    \vskip 2mm
    {\fontsize{16}{0}\selectfont \color{noir}{\oldstylenums{2022}}}

  \end{center}
  \restoregeometry
  \end{titlepage}
  \pagecolor{creme}
  \afterpage{
    \pagecolor{creme}
    \afterpage{\pagecolor{white}}
  }
}
%\def\genre#1{\def\@genre{#1}}
%\def\subtitle#1{\def\@subtitle{#1}}
\makeatother
